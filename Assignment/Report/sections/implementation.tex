\section{MIPS Assembly Implementation}
\subsection{Introduction}
\qquad In this project, we aim to emulate the Battleship game using the \textit{MIPS} assembly language. Due to resource constraints and to keep things simple, we will work with a smaller grid size which is \textbf{7$\times$7}\\

\qquad A ship location is indicated by the
coordinates of the bow and the stern of the ship (row$_{\text{bow}}$, column$_{\text{bow}}$, row$_{\text{stern}}$, column$_{\text{stern}}$).

\subsection{Overall Structure}
Describe the overall structure of the code.\\

Highlight the main sections, such as data section (.data), text section (.text), and various segments like the game menu, rules screen, ship placement, and shooting.

\subsection{Constants and Definitions}
Explain the purpose of constants defined using .eqv (e.g., system calls, characters, and game-related constants).

\subsection{Data Section}
Describe the data section (.data) and the defined constants.\\

Discuss the purpose of player maps, grid size, ship and shot coordinates, and the title/rules strings.

\subsection{User Interface}
Explain how the game menu is displayed to the user.\\

Discuss how the rules screen is presented with ASCII art.

\subsection{Game Logic}
Analyze the game flow from the main function.\\

Discuss how players take turns placing ships and shooting, including input validation.

\subsection{Helper Functions}
Describe the purpose of each function \(e.g., game_menu, rules_screen, read_ship, read_shot\).\\

Highlight the functionality of key functions and their interaction.

\subsection{Input Validation - Error Handling}
Discuss how the code validates user input for ship and shot coordinates. \\

Analyze the error messages and how they guide the user.

\subsection{Gameplay Logic}
Explain the sequence of events during ship placement and shooting.\\

Discuss how ships are drawn on the maps and the conditions for hitting or missing.